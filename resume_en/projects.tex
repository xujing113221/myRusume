%-------------------------------------------------------------------------------
%	SECTION TITLE
%-------------------------------------------------------------------------------
\cvsection{Projects}

%-------------------------------------------------------------------------------
%	CONTENT
%-------------------------------------------------------------------------------
\begin{cventries}
%---------------------------------------------------------
\cventry
{Welfenlab, Leibniz University Hannover} % Job title
{Laboratory 3D Graphic Data Processing in Medicine} % Organization
{Hannover, Germany} % Location
{Aug. 2020 – Sept. 2020} % Date(s)
{
	\begin{cvitems} % Description(s) of tasks/responsibilities
 		\item {Expand a given GUI gradually into a DICOM (Digital Imaging and Communications in Medicine) visualization and segmentation program.}
 		\item {Program in Java with Java3d, read in DICOM files and display images in 3 view modes( transversal, saggital or frontal).}
 		\item{Implement various segmentation algorithms (Min-Max segmentation, Region Grow), and 3D visualization algorithms (Point Cloud, Marching Cube, Texture2d / 3d).}
 	%	\item {More informations: \textsl{\href{http://www.welfenlab.de/fileadmin/forschung/gebiete/YaDiV/docu/YaDiV_first_steps.html}{YaDiV - First Steps}}}
 	\end{cvitems}
}
{Java3d, DICOM, Segment, 3D Visualization,GUI-Design}


%---------------------------------------------------------
\cventry
{Instituts für Regelungstechnik, Leibniz University Hannover} % Job title
{Implement GUI to Backup, Restore and Upload} % Organization
{Hannover, Germany} % Location
{Oct. 2019 – Jan. 2020} % Date(s)
{
	\begin{cvitems} % Description(s) of tasks/responsibilities
		\item {Design a Linux based Graphic User Interface (GUI) with the functions to backup, restore and upload in C++ using software Qt framework, upload and share the codes to the repository by Git to review and revise.}
		\item {Write seminar thesis with LaTex: Implementierung einer grafischen Benutzeroberfläche zum Backup, Restore und Upload}
	\end{cvitems}
	}
{QT, C++, Linux, Git, \LaTeX, GUI Design}

%---------------------------------------------------------
  \cventry
    {China National Innovation and Entrepreneurship Training Project for Students} 
    {Robot Acrobatics Show based on Multi-Axis Aircraft Indoors}
    {Xian, China} % Location
    {Jun. 2015 – Jun. 2016} % Date(s)
    {
      \begin{cvitems} % Description(s) of tasks/responsibilities
        \item {Design a quadcopter to achieve autonomous aerial robot in the indoors environment, which consists of a flight control (using STM32) and an indoor positioning system (IPS). }
        \item {The IPS is built by two infrared cameras and a computer in the lab. By using 3 marks on the quadcopter, the IPS can generate 3D coordinates and the flight attitude of the quadrotor. }
        \item {Programming in C/C++, and image processing in OpenCV. Applied quadratic programming algorithm to generate feasible trajectories for autonomous flight in Matlab and gesture control through Kinect.}
        \item {Won the champion at Xidian University}
        \item {Demo on Youtube:  \url{https://youtu.be/7mPEIL4pCoQ}}
      \end{cvitems}
    }
	{C/C++,  Quadcopter, STM32, Matlab, Image Processing, Control Algorithm }	

%---------------------------------------------------------
  \cventry
    {$27^{th}$ Xinghuo Cup Electronic Competition hosted by Xidian University} % Job title
    {Mini Self Balancing Robot} % Organization
    {Xian, China} % Location
    {Sept. 2015 - Dec. 2015} % Date(s)
    {
      \begin{cvitems} % Description(s) of tasks/responsibilities
        \item {Design and build a two-wheel model which can be controlled remotely. The attitude data, such as angle and position, is collected through the gyroscope accelerometer. The algorithm, Proportional-Integral-Derivative controller (PID), keeps the balance.}
       \item {Additionally, design and build the remote controller with Ardunio, by which the robot can be controlled remotely via wireless connection
       	through joystick and gravity.}
       \item {Won the prize Best Equal (10 from 3000 entries)}
      \end{cvitems}
    }{Control Algorithm,  Embedded System, Remote Control}

%---------------------------------------------------------
\end{cventries}

