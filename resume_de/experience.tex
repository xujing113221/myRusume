%-------------------------------------------------------------------------------
%	SECTION TITLE
%-------------------------------------------------------------------------------
\cvsection{Erfahrungen}


%-------------------------------------------------------------------------------
%	CONTENT
%-------------------------------------------------------------------------------
\begin{cventries}
%---------------------------------------------------------
\cventry
{Instituts für Regelungstechnik, Leibniz Universität Hannover} % Job title
{Implementierung einer grafischen Benutzeroberfläche zum Backup,Restore und Upload} % Organization
{Hannover, Deutschland} % Location
{Oct. 2019 – Jan. 2020} % Date(s)
{
	\begin{cvitems} % Description(s) of tasks/responsibilities
		\item {I Entwurf einer Linux-basierten grafischen Benutzeroberfläche (GUI) mit den Funktionen zum Sichern, Wiederherstellen und Hochladen in C++ unter Verwendung des Software-Qt-Frameworks Qt, Hochladen und Weitergeben der Codes an das Repositorium durch Git zur Überprüfung und Überarbeitung.}
		\item {Seminar Thesis: Implementierung einer grafischen Benutzeroberfläche zum Backup,Restore und Upload}
	\end{cvitems}
	}
{QT, C++, Linux, Git}

%---------------------------------------------------------
  \cventry
    {Nationales chinesisches Projekt zur Ausbildung von Studenten in Innovation und Unternehmertum} 
    {Roboter-Akrobatik-Show basierend auf Multi-Axis Aircraft Indoors}
    {Xian, China} % Location
    {Jun. 2015 – Jun. 2016} % Date(s)
    {
      \begin{cvitems} % Description(s) of tasks/responsibilities
        \item {Entwurf eines Quadrotors zur Realisierung eines autonomen Flugrobot- ers in der Innenumgebung, der aus einer Flugsteuerung (unter Verwendung von STM32) und einem Indoor- Positionierungssystem (IPS) besteht.}
        \item {Das IPS wird von zwei Infrarotkameras und einem Computer im Labor aufgebaut. Mit Hilfe von 3 Markierungen auf dem Quadrotor kann das IPS 3D-Koordinaten und die Fluglage des Quadrotors generieren.}
        \item {Die Programmierung erfolgt in C/C++, die Bildverarbeitung in OpenCV. Angewandter quadratischer Programmieralgorithmus zur Erzeugung realisierbarer Flugbahnen für den autonomen Flug in Matlab und Gestensteuerung durch Kinect.}
        \item {Demo auf Youtube:  \url{https://youtu.be/7mPEIL4pCoQ}}
      \end{cvitems}
    }
	{C/C++, OpenCV, Control algorithm, Quadrotor, STM32 }	

%---------------------------------------------------------
  \cventry
    {Zhejiang University - School of Medicine} % Job title
    {Konstruktion, Hardware- und Embedded-Software-Entwicklung} % Organization
    {Hangzhou, China} % Location
    {Jan.2018-Aprl.2018} % Date(s)
    {
      \begin{cvitems} % Description(s) of tasks/responsibilities
        \item {EntwerfenundbauenSieeinGerätzurBeobachtungdesVerhaltensderexperimentellenMäuse.Esbestehtauseiner mechanischen Konstruktion (Zeichnung in AutoCAD und SolidWorks, Kunststoffteile im 3D-Drucker ausgedruckt, Metallteile in der Fertigung konstruiert), einem Controller mit Arduino zur Steuerung des Schrittmotors und eines Hall-Effekt-Sensors sowie einer steckbaren Kamera, die von der Software gesteuert wird, um die von mir entwickelten Verhaltensweisen im LabView aufzuzeichnen.}
      \end{cvitems}
    }{Labview, Arduino, AutoCAD, SolidWorks}

%---------------------------------------------------------
  \cventry
    {27. Xinghuo Cup Electronic Wettbewerb} % Job title
    {Mini Self Balancing Robot} % Organization
    {Xian, China} % Location
    {Spte.. 2015 - Dec. 2015} % Date(s)
    {
      \begin{cvitems} % Description(s) of tasks/responsibilities
        \item {Entwerfen und bauen Sie ein zweirädriges Modell, das ferngesteuert werden kann. Die Lagedaten, wie Winkel und Position, werden durch den Gyroskop-Beschleunigungsmesser erfasst. Der Algorithmus, Proportional-Integral-Derivativ-Regler (PID), hält das Gleichgewicht. Zusätzlich entwerfen und bauen Sie die Fernsteuerung mit Ardunio, mit der der Roboter über eine drahtlose Verbindung mittels Joystick und Schwerkraft ferngesteuert werden kann.}
        \item {Implemented API server which is communicating with game client and In-App Store, along with two other team members who wrote the game logic and designed game graphics.}
        \item {Won the 2nd prize in final evaluation.}
      \end{cvitems}
    }{PID, Embedded System, Remote Control}

%---------------------------------------------------------
\end{cventries}
