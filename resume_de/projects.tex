%-------------------------------------------------------------------------------
%	SECTION TITLE
%-------------------------------------------------------------------------------
\cvsection{Projekte}


%-------------------------------------------------------------------------------
%	CONTENT
%-------------------------------------------------------------------------------
\begin{cventries}

%---------------------------------------------------------
 \cventry
 {Welfenlab, Leibniz Universität Hannover} % Job title
 {Labor Graphische 3D Datenverarbeitung in der Medizin} % Organization
 {Hannover, Deutschland} % Location
 {Aug. 2020 – Sept. 2020} % Date(s)
 {
 	\begin{cvitems} % Description(s) of tasks/responsibilities
 		\item {Im Labor wird eine vorgegebene GUI schrittweise zu einem DICOM(Digital Imaging and Communications in Medicine) Visualisierungs- und Segmentierungs-Programm erweitert.}
 		\item {Programmiert wird in Java bzw. Java3d. DICOM Dateien einlesen und Bilder anzeigen, Implementierung verschiedener Segmentierungsalgorithmen: Min-Max Segmentierung, Region Grow und 3D-Visualisierungsalgorithmen: Punktwolke, Marching Cube, Texture.}
 		\item {Weitere Einführung: \textsl{\href{http://www.welfenlab.de/fileadmin/forschung/gebiete/YaDiV/docu/YaDiV_first_steps.html}{YaDiV - First Steps}}}
 	\end{cvitems}
 	}
 {Java3d, DICOM, Segmentierung,3D-Visualisierung, GUI-Entwicklung}

%---------------------------------------------------------
\cventry
{Instituts für Regelungstechnik, Leibniz Universität Hannover} % Job title
{Implementierung einer grafischen Benutzeroberfläche zum Backup,Restore und Upload} % Organization
{Hannover, Deutschland} % Location
{Okt. 2019 – Jan. 2020} % Date(s)
{
	\begin{cvitems} % Description(s) of tasks/responsibilities
		\item {Entwurf einer Linux-basierten grafischen Benutzeroberfläche (GUI) mit den Funktionen zum Sichern, Wiederherstellen und Hochladen in C++ unter Verwendung des Software-Qt-Frameworks Qt.}
		\item{Hochladen und Weitergeben der Codes an das Repositorium durch Git zur Überprüfung und Überarbeitung.}
		\item {Schreiben des Bericht in LaTeX.}
	\end{cvitems}
	}
{QT, C++, Linux, Git, \LaTeX, GUI-Entwicklung}

%---------------------------------------------------------
  \cventry
    {Nationales chinesisches Projekt zur Ausbildung von Studenten in Innovation und Unternehmertum} 
    {Roboter-Akrobatik-Show basierend auf Multi-Axis Aircraft Indoors}
    {Xian, China} % Location
    {Jun. 2015 – Jun. 2016} % Date(s)
    {
      \begin{cvitems} % Description(s) of tasks/responsibilities
        \item {Entwurf eines Quadrotors zur Realisierung eines autonomen Flugroboters in der Innenumgebung, der aus einer Flugsteuerung (unter Verwendung von STM32) und einem Indoor- Positionierungssystem (IPS) besteht.}
        \item {Das IPS wird von zwei Infrarotkameras und einem Computer im Labor aufgebaut. Mit Hilfe von 3 Markierungen auf dem Quadcopter kann das IPS 3D-Koordinaten und die Fluglage des Quadcopters generieren.}
        \item {Die Programmierung erfolgt in C/C++, die Bildverarbeitung in OpenCV. Angewandter quadratischer Programmieralgorithmus zur Erzeugung realisierbarer Flugbahnen für den autonomen Flug in Matlab und Gestensteuerung durch Kinect.}
        \item {Gewann den Champion an der Xidian Universität.}
        \item {Demo auf Youtube:  \url{https://youtu.be/7mPEIL4pCoQ}}
      \end{cvitems}
    }
	{C/C++,  Quadcopter, STM32, Kinect, Matlab, Bildverarbeitung, Regelungstechnik }	

%---------------------------------------------------------
  \cventry
    {27. Xinghuo Cup Electronic Wettbewerb, veranstaltet von der Xidian Universität} % Job title
    {Mini Self Balancing Robot} % Organization
    {Xian, China} % Location
    {Sept. 2015 - Dez. 2015} % Date(s)
    {
      \begin{cvitems} % Description(s) of tasks/responsibilities
        \item {Entwerfen und bauen Sie ein zweirädriges Modell, das ferngesteuert werden kann. Die Lagedaten, wie Winkel und Position, werden durch den Gyroskop-Beschleunigungsmesser erfasst. Der Algorithmus, Proportional-Integral-Derivativ-Regler (PID), hält das Gleichgewicht. }
       \item {Zusätzlich entwerfen und bauen Sie die Fernsteuerung mit Ardunio, mit der der Roboter über eine drahtlose Verbindung mittels Joystick und Schwerkraft ferngesteuert werden kann.}
       \item {Gewann den Preis Best Equal (10 von 3000 Einreichungen)}
      \end{cvitems}
    }{Regelungstechnik, Eingebettetes System, Fernsteuerung}

%---------------------------------------------------------
\end{cventries}
