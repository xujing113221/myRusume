%!TEX TS-program = luatex
%!TEX encoding = UTF-8 Unicode
% Awesome CV LaTeX Template for Cover Letter
%
% This template has been downloaded from:
% https://github.com/posquit0/Awesome-CV
%
% Authors:
% Claud D. Park <posquit0.bj@gmail.com>
% Lars Richter <mail@ayeks.de>
%
% Template license:
% CC BY-SA 4.0 (https://creativecommons.org/licenses/by-sa/4.0/)
%


%-------------------------------------------------------------------------------
% CONFIGURATIONS
%-------------------------------------------------------------------------------
% A4 paper size by default, use 'letterpaper' for US letter
\documentclass[11pt, a4paper]{awesome-cv}
%\usepackage[english,ngerman]{babel}
\usepackage[ddmmyyyy]{datetime}
\renewcommand{\dateseparator}{.}
\usepackage{fontawesome}
% Configure page margins with geometry
\geometry{left=1.4cm, top=.8cm, right=1.4cm, bottom=1.8cm, footskip=.5cm}

% Specify the location of the included fonts
\fontdir[fonts/]



% Color for highlights
% Awesome Colors: awesome-emerald, awesome-skyblue, awesome-red, awesome-pink, awesome-orange
%                 awesome-nephritis, awesome-concrete, awesome-darknight
\colorlet{awesome}{awesome-red}
% Uncomment if you would like to specify your own color
% \definecolor{awesome}{HTML}{CA63A8}

% Colors for text
% Uncomment if you would like to specify your own color
% \definecolor{darktext}{HTML}{414141}
% \definecolor{text}{HTML}{333333}
 \definecolor{graytext}{HTML}{222222}
 \definecolor{lighttext}{HTML}{5D5D5D}

% Set false if you don't want to highlight section with awesome color
\setbool{acvSectionColorHighlight}{true}

% If you would like to change the social information separator from a pipe (|) to something else
\renewcommand{\acvHeaderSocialSep}{\quad\textbar\quad}

%-------------------------------------------------------------------------------
%	PERSONAL INFORMATION
%	Comment any of the lines below if they are not required
%-------------------------------------------------------------------------------
% Available options: circle|rectangle,edge/noedge,left/right
%\photo[circle,noedge,left]{./examples/profile}
\name{Jing}{Xu}
%\position{Software Architect{\enskip\cdotp\enskip}Security Expert}
\address{Nordfelder Reihe 4C, 30159, Hannover, Deutschland}

\mobile{(+49) 15202178266}
\email{jing.xu@stud.uni-hannover.de}
\homepage{xujing113221.github.io}
\github{xujing113221}
% \gitlab{gitlab-id}
% \stackoverflow{SO-id}{SO-name}
% \twitter{@twit}
% \skype{skype-id}
% \reddit{reddit-id}
% \medium{madium-id}
% \googlescholar{googlescholar-id}{name-to-display}
%% \firstname and \lastname will be used
% \googlescholar{googlescholar-id}{}
% \extrainfo{extra informations}

%\quote{``Be the change that you want to see in the world."}


%-------------------------------------------------------------------------------
%	LETTER INFORMATION
%	All of the below lines must be filled out
%-------------------------------------------------------------------------------
\providecommand{\recname}{}
\providecommand{\companyname}{Google Inc.}
\providecommand{\companyaddress}{Helstorfel Str. 23}
\providecommand{\companylocal}{}
\providecommand{\applicatename}{XXXXX (Nr. xxx)}
\providecommand{\letterbegin}{Sehr geehrte Damen und Herren,}
% The company being applied to
\recipient
  {\recname}
  {\companyname \\ \companyaddress \\ \companylocal}
% The date on the letter, default is the date of compilation
\letterdate{\today}
% The title of the letter
\lettertitle{Bewerbung für die Praktikumsstelle als \applicatename}
% How the letter is opened
\letteropening{\letterbegin}
% How the letter is closed
\letterclosing{Mit freundlichen Grüßen}
% Any enclosures with the letter
\letterenclosure[i.A.]{Lebenslauf\\ Bescheinigungen \\Notenspiegel \\Zeugnissen\\Überblick über meine technischen Innovationen }


%-------------------------------------------------------------------------------
\begin{document}

% Print the header with above personal informations
% Give optional argument to change alignment(C: center, L: left, R: right)
\makecvheader[R]

% Print the footer with 3 arguments(<left>, <center>, <right>)
% Leave any of these blank if they are not needed
\makecvfooter
  {}
  {Jing Xu~~~·~~~Anschreiben}
  {}

% Print the title with above letter informations
\makelettertitle

%-------------------------------------------------------------------------------
%	LETTER CONTENT
%-------------------------------------------------------------------------------
\begin{cvletter}


ich bin sehr interessiert an der Praktikumsstelle als \applicatename und reiche deshalb meine Bewerbung ein.

Ich studiere jetzt im 3. Semester an der Leibniz Universität Hannover (LUH) den Masterstudiengang "Elektrotechnik und Informationstechnik" mit dem Schwerpunkt "Computer Engineering". Im Jahr 2017 schloss ich erfolgreich den Bachelor Science in Biomedizintechnik an der Xidian Universität in China ab.

Als ich 2013 mit meinem Bachelor begann, nahm ich am 25. Xinghuo-Cup-Wettbewerb für Elektronik an der Xidian Universität teil, was mein großes Interesse an der praktischen Umsetzung von Elektronik- und Computertechnik hervorrief. Es erregte meinen Wunsch, mich auch in Deutschland weiterzuentwickeln.

Um mich in diesem Bereich der Elektronik und Informatik weiterzuentwickeln, habe ich mir parallel zum Hauptstudium das Wissen, das Know-how und die technischen Werkzeuge (mechanische Konstruktion, Hard- und Softwareentwicklung) selbst angeeignet.

Während meines vierjährigen Bachelorstudiums nahm ich an sieben elektronischen Wettbewerben mit mehr als zehn Einreichungen meiner technischen Innovationen teil. Alle Einreichungen gewannen den Preis (zwei davon gewannen den nationalen Preis). Auch sammelte ich so die ersten Erfahrungen, um die technischen Theorien auf die praktischen Geräte umzusetzen.

Sie können mich als schnell lernende, hochinnovative Person kennen lernen, die mit bewährten Fähigkeiten in der Software- und Hardware-Entwicklung gut in der praktischen Umsetzung ist. Mit großem Interesse an Elektronik und Informatik bin ich begeistert, die zukunftsweisenden Technologien zu beherrschen.

Ich weiß Ihre Zeit und Ihre Bemühungen zu schätzen, meine Bewerbung zu prüfen, und ich freue mich auf die Gelegenheit, mich ausführlicher vorzustellen.\\\\

%\lettersection{Why Google?}

%\lettersection{Why Me?}

\end{cvletter}


%-------------------------------------------------------------------------------
% Print the signature and enclosures with above letter informations
\makeletterclosing

\end{document}
